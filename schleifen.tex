\RequirePackage{pdfmanagement-testphase}
\DeclareDocumentMetadata{pdfstandard=A-2b}
\documentclass[a5paper,landscape,ngerman,10pt]{article}
\usepackage{babel,listings,geometry,relsize,unicode-math}
%\setmainfont{Libertinus Serif}
%\setmathfont{Libertinus Math}
\setmainfont{TeX Gyre Pagella}
\setmathfont{TeX Gyre Pagella Math}
\lstloadlanguages{[11]C++, Modula-2, Python}
\title{Schleifen wegoptimieren}
\author{Timm Knape}
\setlength{\parindent}{0em}
\setlength{\parskip}{1em}
\begin{document}
\newcommand{\cpp}{C\kern-.1em\raise.30ex\hbox{\smaller{++}}}
\lstset{
  columns=fullflexible,
  language=[11]C++,
  mathescape=true,
  escapechar=\&
}
%
\maketitle
%
\section{Aufgabenstellung}
%
Während meines Informatik-Studiums haben wir die
Programmiersprache \hbox{\textsc{Modula-$2$}} gelernt.
Mühevoll zogen sich die Wochen dahin, in denen wir die
Grundstrukturen verdauten.
Irgendwann wurden Schleifen eingeführt und es kam auf
einem Übungszettel zu folgender Aufgabe:

\begin{quotation}
\textit{Schreibe eine Funktion, welche die 
Summe der ersten $n$ ganzen Zahlen liefert.}
\end{quotation}

Erwartet war irgendetwas in der Form

\begin{quotation}
\begin{lstlisting}[language=Modula-2, morekeywords={TO}]
PROCEDURE Sum&\,&($n$: INTEGER): INTEGER;
   VAR $i,s$: INTEGER;
BEGIN
   $s := 0$;
   FOR $i := 1$ TO $n$ DO
      $s := s + 1$;
   END;
   RETURN $s$;
END Sum;
\end{lstlisting}
\end{quotation}

In \cpp\ kann die Funktion etwa so aussehen:

\begin{quotation}
\begin{lstlisting}
int sum&\,&(int $n$) {
   int $s$ {$\,0\,$};
   for (int $i$ {$\,1\,$}; $i$ <= $n$; ++$i$) {
      $s$ += $i$;
   }
   return $i$;
}
\end{lstlisting}
\end{quotation}

Bleiben wir für den Rest dieses Beitrags in dieser Sprache.

\section{Gauß-Methode}

Ich war natürlich vorwitzig und sah nicht ein,
nur zur Übung eine Schleife zu verwenden,
wenn es nach einem Satz von 
Carl~Friedrich~Gauß auch viel eleganter geht:

\begin{quotation}
\begin{lstlisting}
int sum&\,&(int $n$) {
   return $n * (n + 1) / 2$;
}
\end{lstlisting}
\end{quotation}

Der junge Gauß hat der Legende nach in der Schule die
Strafarbeit erhalten, die ersten hundert Zahlen
zusammenzuaddieren.
Er war jedoch nach kurzer Zeit fertig.

Er schrieb die Zahlen einmal vorwärts und einmal rückwärts auf:

\begin{quotation}
\begin{tabular}{c|c|c|c|c|c}
$1$&$2$&$3$&\ldots&$n-1$&$n$\\
\hline
$n$&$n-1$&$n-2$&\ldots&$2$&$1$
\end{tabular}
\end{quotation}

Die Summe der Spalten ist stets $n+1$.
Die Summe aller Zahlen in der Tabelle beträgt also $n\cdot(n+1)$.
Da die Zahlen aber zweimal aufgeschrieben wurden, muss das
Ergebnis noch durch zwei geteilt werden.

\section{Überlauf}

Ich habe auf diese Aufgabe die vollen Punkte bekommen.
Inzwischen frage ich mich jedoch, ob dies gerechtfertigt war:
Kann es nicht sein, dass durch einen Überlauf falsche Werte
berechnet werden, die in einer Schleife langsam, aber korrekt,
berechnet worden wären?

In vielen Sprachen hat eine Integer-Variable nur eine feste
Länge.
Auf meinem System sind das $32$ Bit.
Wenn das Produkt zweier Zahlen größer oder gleich $2^{32}$
wird, dann wird das Ergebnis abgeschnitten.
Schon wenn das Ergebnis $2^{31}$ erreicht, entstehen Probleme:
Das oberste Bit ist für das Vorzeichen reserviert.
Das Ergebnis wird als negative Zahl angezeigt, wenn das
höchste Bit gesetzt ist.

Natürlich klappt auch nur die Lösung mit der Schleife in einem
bestimmten Werte-Bereich.
Dieser ist jedoch viel größer.
Bei der Schleife muss nur das Ergebnis kleiner $2^{31}$ sein,
also

\[2^{31} > \frac{n\cdot(n + 1)}2\]

Durch Multiplizieren mit $2$ ergibt sich

\[2^{32} > n\cdot(n+1)\]

Die Ungleichung bleibt bestehen, wenn das $n+1$ durch $n$
ersetzt wird

\[2^{32} > n^2\]

Und durch Ziehen der Quadrat-Wurzel ergibt sich

\[2^{16} > n\]

Und diese Grenze ist scharf! Schon mit $n=2^{16}$ ist
$n\cdot(n+1)$ größer als $32$ Bit.
Auch mit der Schleife können wir nur bis zur Zahl $65.535$
aufsummieren, die gerade noch in $16$ Bit hineinpasst.
Bei größeren Zahlen gibt es einen Überlauf im Ergebnis.

Bei der trivialen Umsetzung der Gauß-Methode tritt der
Überlauf schon viel früher ein.
Schon wenn $n$ mit $n+1$ multipliziert wird, kann ein
Überlauf entstehen, wenn nicht

\[2^{31}>n\cdot(n+1)\]

gilt.
Durch die gleiche Abschätzung ergibt sich

\[2^{31}>n\]

und weiter durch die Wurzel

\[\sqrt{2}\cdot2^{15}>46.340\geq n\]

Schon mit $n=46.341$ passt das Ergebnis nicht mehr in $31$ Bit.

\section{Gauß ohne Überlauf}

Der Überlauf kann verhindert werden, wenn wir entweder $n$ oder
$n+1$ erst durch $2$ teilen und dann mit dem anderen Faktor
multiplizieren.
Jedoch dürfen wir nur die Zahl durch $2$ teilen, die gerade
ist.
Andernfalls entstehen Rundungsfehler.

Eine erste Lösung ist eine Fallunterscheidung:

\begin{quotation}
\begin{lstlisting}
int sum&\,&(int $n$) {
   if ($n\mathrel\% 2$) {
      return $(n + 1)/2\:*\:n$;
   } else {
      return $n/2 \:*\: (n + 1)$;
   }
}
\end{lstlisting}
\end{quotation}

Damit funktioniert Gauß wieder bis zur Grenze $65.535$.
Und erst damit ist die Aufgabe eigentlich gelöst.

Der Compiler wird hoffentlich die Modulo-Operation durch eine
Und-Verknüpfung und die Division durch eine
Verschiebe-Operation ersetzen, die auf den meisten
Prozessoren deutlich weniger Takte benötigen (und damit
schneller sind).

Wenn man dem Compiler nicht traut, kann man auch direkt
auf Bit-Ebene die Anweisungen geben:

\begin{quotation}
\begin{lstlisting}
int sum&\,&(int $n$) {
   if ($n \mathrel\& 1$) {
      return $((n + 1) >\!> 1) * n$;
   } else {
      return $(n >\!> 1) * (n + 1)$;
   }
}
\end{lstlisting}
\end{quotation}

Aber es geht \textit{noch\/} besser.
Die Fallunterscheidung macht bei aktuellen Prozessor-Pipelines
Probleme.
Wenn der Prozessor den falschen Weg rät, muss die
Pipeline neu angekurbelt werden.
Besser und effizienter sind Lösungen, die keine
Sprünge benötigen.
Und die gibt es.

Hier möchte ich eine Low-Level Variante vorstellen.
Wir wissen, dass $n$ nicht größer als $65.535$ werden
kann.
Dieses $n$ passt aber noch in $16$ Bit.
Dann passt das Produkt von $n$ und $n+1$ aber noch in
$32$ Bit.
Nicht in $31$ Bit, aber in $32$!
Somit klappt alles, wenn wir mit Zahlen rechnen, die
keinen negativen Wertebereich haben.
Hier der Vorschlag:

\begin{quotation}
\begin{lstlisting}
inline int sum&\,&(int $n$) {
   auto $\mathit{un}$ {&\,&static_cast<unsigned>&\,&($n$)$\,$};
   return static_cast<int>&\,&(
      $(\mathit{un} * (\mathit{un} + 1)) >\!> 1$
   );
}
\end{lstlisting}
\end{quotation}

Der Wechsel nach \lstinline!unsigned! ist notwendig, damit
bei der Verschiebung immer $0$-Bits nachgezogen werden.
Bei \lstinline!int! bleibt das oberste Bit unberücksichtigt
(und dadurch bleibt die Zahl negativ).

Sieht gewaltig aus (mit dem \lstinline!static_cast!), aber
der generierte Code ist sehr effizient.
Es müssen nur andere Maschinenbefehle verwendet werden.

\section{Andere Programmiersprachen}

Viele andere Sprachen haben das gleiche Problem:

\begin{enumerate}
 \item Java,
 \item C\#,
 \item Rust.
 \item Go.
\end{enumerate}

Es gibt aber Sprachen, die intern mit beliebig großen
Zahlen rechnen können:

\begin{enumerate}
 \item Scheme/Lisp und
 \item Python,
 \item Ruby.
\end{enumerate}

Unter Python können wir einfach die Standard-Version
des Gauß-Algorithmus verwenden:

\begin{quotation}
\begin{lstlisting}[language=Python]
def sum&\,&($n$):
   return $n * (n + 1) \mathrel{/\!/} 2$
\end{lstlisting}
\end{quotation}

Jedoch ist der Umgang mit beliebig großen ganze Zahlen
ein erheblicher Mehraufwand für den Rechner.
Anstatt direkt in eimen Register zu rechnen, müssen die
Zahlen über mehrere Worte des Arbeitsspeichers verteilt
oder in mehreren Registern abgelegt werden.
Dabei entstehen zwangsweise Sprünge, da von vorne herein
nicht klar ist, wie viel Speicher die Zahl belegt.

\section{Zusammenfassung}

Selbst bei so einer einfachen Schleifen-Optimierung ist
Vorsicht angesagt.
Die enorme Kosteneinsparung kann ggf.\ eine Einschränkung
des Anwendungsbereichs zur Folge haben.
Wenn man dies nicht beachtet, produziert der verbesserte
Code schwer zu findende Fehler.
Mit etwas Nachdenken finden sich aber häufig verbesserte
Lösungen.

Nicht jeder Code muss bis zu seinem Maximum optimiert
werden.
Schon die Beschreibung des Maximums ist nicht leicht.
Soll der Code möglichst schnell laufen?
Oder möglichst klein sein?
Beides schließt sich oft aus.
Schön ist es, wenn wie in diesem Beispiel eine kleine,
schnelle Lösung existiert.

Leider rentiert es sich oft nicht, eine solche Lösung
zu suchen.
Die Rechner sind schnell genug, dass auch suboptimale
Lösungen verwendet werden können.
Diese verbrauchen jedoch mehr Energie, Speicher und
Rechenzeit als eine bessere Lösung, über die man ein
klein wenig mehr nachgedacht hat.

\section{Weiter denken}

JavaScript verwendet nur $64$-Bit Floating-Point Zahlen.
Wie weit kann die Schleifen-Variante korrekte Ergebnisse
liefern? Wie weit der einfache Gauß?

Wie verhalten sich Systeme, auf denen $64$-Bit Integer-Zahlen
verwendet werden können?

Wie verhalten sich Systeme, die direkt mit
\lstinline!unsigned! Zahlen rechnen?
Warum kann die Shift-Variante dort nicht verwendet werden?

Ein ähnliches Problem tritt beim Berechnen eines Mittelwerts
auf:

\[\frac{a+b}2\]

Wie verhalten sich hier die Grenzen?

Kann auch eine Shift-Variante entwickelt werden, welche
das zusätzliche Vorzeichen-Bit ausnutzt?

\end{document}
